\documentclass{article} % Try also "scrartcl" or "paper"
\usepackage{blindtext} % for dummy text
\usepackage{hyperref}
\usepackage{pdflscape}
\usepackage[margin=1in]{geometry}
% \usepackage[margin=2cm]{geometry}   % to change margins
% \usepackage{titling}             % Uncomment both to   
% \setlength{\droptitle}{-2cm}     % change title position 
\author{Benjamin S. Knight}
\title{%\vspace{-1.5cm}            % Another way to do
Leveraging Snowplow Event Tracking for \\ Better Understanding Visitor Conversion}
\begin{document}
\maketitle
\section{Background}
\indent\indent The central goal of this research is marketing analytics.
\href{http://snowplowanalytics.com/}{Snowplow} is a web event tracker capable of handling tens of millions of events per day. Using this data, we hope to answer the question of how different visitor experiences at a company's marketing site relate to the probability of those visitors ultimately becoming paying customers. \\
\indent Applying machine learning to event data gathered from web applications is becoming standard practice.\footnote{Falchuk, Mesterharm, and Panagos. 2016. (\url{http://www.research.rutgers.edu/~mesterha/emerging_2016_3_40_50024.pdf})} However, here we are trying to make inferences about customers and potential customers using information attained from before they first used the application. In fact, the overwhelming majority of observations come from users who have never used the application at all. \\
\indent The Snowplow data we have at our disposal can be thought of as a much larger, richer version of the MSNBC.com Anonymous Web Data Data Set hosted by the University of California, Irvine's Machine Learning Repository.\footnote{Heckerman, David(\url{https://archive.ics.uci.edu/ml/datasets/MSNBC.com+Anonymous+Web+Data})} Like the MSNBC.com dataset, we have access to page view events along with other kinds of events. Unlike the MSNBC.com data set, we can map these events to individual cookies, then map the cookies to accounts of application users. \\
\indent This sort of information can be useful in determining what portions of the marketing site should receive priority with respect to A/B testing, new content, and so forth. At the same time, having awareness of visitors with a higher than normal likelihood of becoming customers would help the Sales Department better utilize scarce resources.

\section{Problem Statement}
\indent\indent To what extent can we infer a visitor's likelihood of becoming a paying customer based upon that visitor's experience on the company marketing site? Assuming that the predicted visitor behavior is superior to blind guessing, what specific factors (both within and outside of the company's control) contribute to a visitor's likelihood of becoming a paying customer?\\
\indent In more concrete terms, we are essentially confronted with a binary classification problem - will the account in question add a credit card (cc\_date\_added IS NOT NULL `yes'/`no')? This labeling information is contained in the `cc' column within the file `munged\_df.csv.' 

\section{Datasets and Inputs}
\indent\indent The raw Snowplow data available to us is ~15 gigabytes spanning ~300 variables and tens of millions of events from November 2015 to January 2017. When we omit fields that are not in active use, are redundant, contain personal identifiable information (P.I.I.), or which cannot have any conceivable baring on customer conversion, then we are left with 14.6 million events and the variables shown in Table 1.\\
\indent I use the phrase `variable' as opposed to feature, since this dataset will need to undergo substantial transformation before we can employ any supervised learning technique. Each row has an `event\_id' along with an `event\_name' and a `page\_url.' The event\_id is the row's unique identifier, the event\_name is the type of event, and the page\_url is the URL within the marketing site where the event took place. \\
\indent In transforming the data, we will need to create features by creating combinations of event types and distinct URLs, and counting the number of occurrences while grouping on accounts. For instance, if `.../payment\_plan.com' is a frequent page url, then the number of page views on payment\_plan.com would be one feature, the number of page pings would be another, as would the number of web forms submitted, and so forth. Given that there are six distinct event types and dozens of URLs within the marketing site, then the feature space will likely be in the hundreds of features. This feature space will only widen as we add additional variables to the mix including geo\_region, number of visitors per account, and so forth.\\
\indent At the same time, we will need to filter the data. Since we are interested in the causal relationship between visitors' marketing site experiences and whether they ultimately became paying customers, we can and should omit all events that occur after the time-stamp `cc\_date\_added' - the date when a customer first added a credit card to their account.
\begin{landscape}
\begin{table}[h!]
\centering
\begin{tabular}{| l | l |} 
\hline
Snowplow Variable Name  & Snowplow Variable Description  \\ [0.5ex] 
\hline\hline
event\_id   & The unique Snowplow event identifier \\
\hline
account\_id & The account number if an account is associated with the domain\_userid     \\ 
\hline
reg\_date & The date an account was registered  \\
\hline
cc\_date\_added & The date a credit card was added \\
\hline
collector\_tstamp & The timestamp (in UTC) when the Snowplow collector first recorded the event  \\
\hline
domain\_userid & This corresponds to a Snowplow cookie and will tend to correspond to a single internet device  \\ 
\hline
domain\_sessionidx & The number of sessions to date that the domain\_userid has been tracked \\
\hline
domain\_sessionid & The unique identifier for the Snowplow cookie/session \\
\hline
event\_name & The type of event recorded  \\
\hline
geo\_country & The ISO 3166-1 code for the country that the visitor's IP address is located \\
\hline
geo\_region\_name & The ISO-3166-2 code for country region that the visitor's IP address is in \\
\hline
geo\_city & The city the visitor's IP address is in  \\
\hline
page\_url & The page URL \\
\hline
page\_referrer & The URL of the referrer (previous page) \\
\hline
mkt\_medium & The type of traffic source (e.g. 'cpc', 'affiliate', 'organic', 'social')  \\
\hline
mkt\_source & The company / website where the traffic came from (e.g. 'Google', 'Facebook' \\
\hline
se\_category & The event type  \\ 
\hline                               
se\_action & The action performed / event name (e.g. 'add-to-basket', 'play-video') \\
\hline
br\_name   & The name of the visitor's browser       \\ 
\hline                               
os\_name     & The name of the vistor's operating system       \\   
\hline               
os\_timezone  & The client's operating system timezone       \\ 
\hline        
dvce\_ismobile  & Is the device mobile? (1 = 'yes')  \\
\hline
\end{tabular}
\caption{Snowplow Variables Pre-Transformation}
\label{table:1}
\end{table}
\end{landscape}

\indent The transformed data set is viewable in munged\_df.csv is approximately 38 MB. It contains 290 features that represent counts of various combinations of web events and URLs grouped by account\_id. Next there are two aggregated features - the total number of distinct cookies associated with the account and the sum total of all Internet sessions linked to that account. Next, there are 151 features that represent counts of page view events linked to IP addresses within a certain country (e.g. a count of page views from China, a count of page views from France, and so forth). Next, there are 46 features that represent counts of page views coming from a specific marketing medium (`mkt\_medium'). Recall that `mkt\_medium' is the type of traffic. Examples include `partner\_link,' `adroll,' or `appstore.'
                                                              
\section{Solution Statement}
\indent\indent How reliably can we predict conversion from visitor to paying customer. To this end, I will first experiment with SVMs using a RBF kernel. Other approaches may ultimately prove to be more successful, but a RBF kernel is likely to serve us well as a starting point \footnote{https://arxiv.org/pdf/1606.00930v1.pdf} \\

\section{Benchmark Model}
\indent\indent For a \textit{very} rough baseline of our future model's performance, we can divide the number of accounts where credit card was added (cc\_date\_added) by the number of total accounts. Thus, for our highly imbalanced data the probability of a label denoting a successful customer conversion (cc = 1) is 0.06. The imbalanced nature of the data means that accuracy is not particularly helpful a metric. Instead, I use the Area Under the Curve (AUC) of a Precision-Recall plot. Applying the data post-transformation (see munged\_df.csv) directly to a SVM with RBF kernel and all default settings yielded an AUC of 0.22 - the benchmark model for this project. 

\section{Evaluation Metrics}
\indent\indent Given that positive conversion events are extremely rare (0.4\%), a precision-recall curve is more appropriate in this context compared to the Receiver Operator Characteristic (ROC) curve.\footnote{http://ftp.cs.wisc.edu/machine-learning/shavlik-group/davis.icml06.pdf}\\

\end{document}

